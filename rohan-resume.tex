%%%%%%
% Rohan Garg - curriculum vitae
% Edited using kile
% Uses LaTeX and moderncv
%%%%%%


\documentclass[11pt,a4paper]{moderncv}
\usepackage{helvet}
% moderncv themes
\moderncvtheme[blue]{classic}                 % optional argument are 'blue' (default), 'orange', 'red', 'green', 'grey' and 'roman' (for roman fonts, instead of sans serif fonts)
%\moderncvtheme[green]{classic}                % idem

% character encoding
\usepackage[utf8]{inputenc}                   % replace by the encoding you are using
\usepackage{hyperref}


% adjust the page margins
\usepackage[scale=0.8]{geometry}
%\setlength{\hintscolumnwidth}{3cm}						% if you want to change the width of the column with the dates
\AtBeginDocument{\setlength{\maketitlenamewidth}{10cm}} % only for the classic theme, if you want to change the width of your name placeholder (to leave more space for your address details
\AtBeginDocument{\recomputelengths}                     % required when changes are made to page layout lengths

% personal data
\firstname{Rohan}
\familyname{Garg}
\title{Resumé}
\address{Barcelona, Spain}{}    % optional, remove the line if not wanted
\mobile{+34-625112103}                    % optional, remove the line if not wanted
\email{rohan@garg.io} % optional, remove the line if not wanted

%----------------------------------------------------------------------------------
%            content
%----------------------------------------------------------------------------------
\begin{document}
\maketitle

\section{Education}
\cventry {2008--2012}{Bachelor of Technology}{Maharshi Dayanand University}{Gurgaon}{Electronics and Communication}{}
\cventry {2006--2008}{C.B.S.E. Senior Secondary}{D.A.V. Public School}{Gurgaon}{}{}
\cventry {2005--2006}{C.B.S.E. Secondary}{D.A.V. Public School}{Gurgaon}{}{}

\section{Technical Skills}
\cvcomputer
{\textbf{Languages}}{Ruby, C, C++, Python, \& Bash}
{}{}
\cvcomputer
{\textbf{Frameworks}}{Chef, Docker, AWS, Amazon S3, Aptly, KDE, Qt}
{}{}

\cvcomputer
{\textbf{CI}}{Jenkins}
{}{}


% --------------------------------------
% WORK EX
% --------------------------------------

\section{Work Experience and Positions}
\cventry{August 2012 -- Current}{Debian, Ubuntu}{Backend \& Automation Engineer}{Blue Systems}{}{The bulk of my work at Blue Systems consists of backend engineering and automation. I have been tasked with writing new features for and maintaining a CI instance that builds Debian packages for KDE. During the course of my employment, I have built and successfully shipped multiple products, the most notable being KDE Neon and the Netrunner family of distributions.}{}
\cventry{October 2011 -- August 2012}{libnice}{Intern}{Collabora}{}{libnice is an implementation of the IETF's Interactive Connectivity Establishment (ICE) standard (RFC 5245) and the Session Traversal Utilities for NAT (STUN) standard (RFC 5389). Working on implementing dribble mode in libnice which allows for faster NAT Traversal}
\cventry{May 2011 -- August 2011}{Google Summer of Code 2011}{Student Contractor}{Google}{}{Google Summer of Code is a global program that offers student developers stipends to write code for various open source software projects. It is a highly competitive summer job, where students are required to contribute real world open source projects. Their contributions are released under the open source licenses so that they may benefit everyone. I developed a GUI frontend for SyncEvolution using the KDE and Qt Frameworks}

\subsection{F/OSS Contributions}
\cventry{March 2016 -- Current}{aptly-api}{Co-Maintainer}{Ruby Gem}{\href{https://github.com/KDEJewellers/aptly-api}{Github}}{Aptly is a Debian repository management tool written in Go. aptly-api provides a thin wrapper to interact with Aptly API}
\cventry{March 2015 -- January 2016}{ARM Image Generator}{Author}{Tool to generate images for ARM devices}{\href{https://github.com/netrunner-odroid/arm-image-generator}{Github}}{ARM Image generator aimed to replace Linaro's HWPacks due to it's various limitations and allowed for a very descriptive way of building ARM images for practically any ARM board. Nearly all the functionality of ARM Image generator has now been merged into the live-build tool from Debian}
\cventry{December 2013 -- Current}{Driver Manager KCM}{Author}{App to install proprietary drivers}{\href{https://cgit.kde.org/scratch/garg/kcm-driver-manager.git/}{KDE}}{Driver Manager is a KDE tool that figures out which proprietary drivers are best for your system and allows you to install the one you prefer.}
\cventry{April 2011 -- August 2012}{Telepathy KDE}{Community Developer}{KDE}{Bug fixes, Ux improvements}{Real time Communication has traditionally been a detached feature of Desktop Computing, provided via stand-alone Instant Messaging clients with poor integration into the desktop experience. One of the primary goals of the KDE 4 series is to tighten integration between different components of the environment. The Realtime Communication and Collaboration (RTCC) project aims to tackle just this.}
\cventry{September 2010 -- January 2012}{Project Neon}{Maintainer}{Kubuntu}{Revived Project Neon for Kubuntu along with 2 other developers making use of Launchpad's new feature for daily builds}{Project Neon is a nightly build of the latest KDE trunk. It is an easy way for new contributors to KDE to get started without having to build the entire KDE-SVN tree and maintain the checkout. Additionally, dependencies are automatically handled and updated. This is suitable for new developers, translators, usability designers, documenters, promoters, bug triagers etc}%\href{https://launchpad.net/~neon}{\textit{Project Neon at Launchpad}}}
\cventry{August 2008 -- July 2015}{Kubuntu}{Community Developer}{}{KDE Packaging, fixing build failiures, licensing, \& bugs}{Kubuntu is an operating system built by a worldwide team of expert developers. It contains all the applications you need: a web browser, an office suite, media apps, an instant messaging client and many more.}%\href{https://launchpad.net/~rohangarg}{\textit{My Launchpad page}}}

\section{Talks and Travels}
\cventry{July 2015}{Akademy 2016}{Brno, Czech Republic}{}{}{Continuous Debian Package Delivery -- A peek into tooling and QA measures to provide KDE packages from git using Jenkins.}
\cventry{October 2011}{Google Summer of Code Documentation Sprint}{Googleplex, California}{}{}{Authored a book that guides the reader through the steps of contributing code back to KDE}
\cventry{August 2011}{The Desktop Summit}{Berlin}{}{}{}
\cventry{March 2011}{Project Neon - Make your PC dazzle with neon lights}{conf.kde.in, Bangalore}{}{}{How to use Project Neon to develop KDE/Qt applications}
\cventry{October 2010}{Lightning Talk - Project Neon \& Launchpad}{UDS-N, Florida}{}{}{A talk on how Project Neon is utilizing Launchpad daily build recipes to build trunk KDE Packages}

%---------------------------------------
% STUFF 
%---------------------------------------

\section{Languages}
\cvlanguage{Hindi}{Native}{}
\cvlanguage{English}{Fluent}{}
\cvlanguage{Spanish}{A1 Level}{}
\end{document}
